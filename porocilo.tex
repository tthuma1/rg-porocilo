\documentclass[a4paper,12pt]{article}
%\documentclass[a4paper, 12pt, draft]{book}  Nalogo preverite tudi z opcijo draft, ki vam bo pokazala, katere vrstice so predolge!

\usepackage[utf8x]{inputenc}   % omogoča uporabo slovenskih črk kodiranih v formatu UTF-8
\usepackage[slovene,english]{babel}    % naloži, med drugim, slovenske delilne vzorce
\usepackage[pdftex]{graphicx}  % omogoča vlaganje slik različnih formatov
\usepackage{fancyhdr}          % poskrbi, na primer, za glave strani
\usepackage{amssymb}           % dodatni simboli
\usepackage{amsmath}           % eqref, npr.
%\usepackage{hyperxmp}
\usepackage[hyphens]{url}  % dodal Solina
\usepackage{comment}       % dodal Solina

\usepackage[pdftex, colorlinks=true,
						citecolor=black, filecolor=black, 
						linkcolor=black, urlcolor=black,
						pagebackref=false, 
						pdfproducer={LaTeX}, pdfcreator={LaTeX}, hidelinks]{hyperref}

\usepackage{color}       % dodal Solina
\usepackage{soul}       % dodal Solina

% Use the graphicx package for image inclusion
\usepackage{graphicx}

% Page setup
\usepackage{geometry}

% \geometry{top=2.5cm, bottom=2.5cm, left=2.5cm, right=2.5cm}

% Title and author setup
\title{\vspace{-4cm}Seminarska naloga za Računalniško grafiko\\[0.5cm] \huge \textbf{Frodo's Nightmares} \vspace{-1cm}}
\author{}
\date{}


%%%%%%%%%%%%%%%%%%%%%%%%%%%%%%%%%%%%%%%%
% postavitev strani
%%%%%%%%%%%%%%%%%%%%%%%%%%%%%%%%%%%%%%%%  

% \addtolength{\marginparwidth}{-20pt} % robovi za tisk
% \addtolength{\oddsidemargin}{40pt}
% \addtolength{\evensidemargin}{-40pt}

% \renewcommand{\baselinestretch}{1.3} % ustrezen razmik med vrsticami
% \setlength{\headheight}{15pt}        % potreben prostor na vrhu
% \renewcommand{\chaptermark}[1]%
% {\markboth{\MakeUppercase{\thechapter.\ #1}}{}} \renewcommand{\sectionmark}[1]%
% {\markright{\MakeUppercase{\thesection.\ #1}}} \renewcommand{\headrulewidth}{0.5pt} \renewcommand{\footrulewidth}{0pt}
\fancyhf{}
\fancyhead[LE,RO]{\sl \thepage} 
%\fancyhead[LO]{\sl \rightmark} \fancyhead[RE]{\sl \leftmark}
\fancyhead[RE]{\sc \tauthor}              % dodal Solina
\fancyhead[LO]{\sc Diplomska naloga}     % dodal Solina


\newcommand{\BibTeX}{{\sc Bib}\TeX}

%%%%%%%%%%%%%%%%%%%%%%%%%%%%%%%%%%%%%%%%
% naslovi
%%%%%%%%%%%%%%%%%%%%%%%%%%%%%%%%%%%%%%%%  

\newcommand{\autfont}{\Large}
\newcommand{\titfont}{\LARGE\bf}
\newcommand{\clearemptydoublepage}{\newpage{\pagestyle{empty}\cleardoublepage}}
\setcounter{tocdepth}{1}	      % globina kazala

%%%%%%%%%%%%%%%%%%%%%%%%%%%%%%%%%%%%%%%%
% konstrukti
%%%%%%%%%%%%%%%%%%%%%%%%%%%%%%%%%%%%%%%%  

%%%%%%%%%%%%%%%%%%%%%%%%%%%%%%%%%%%%%%%%%%%%%%%%%%%%%%%%%%%%%%%%%%%%%%%%%%%%%%%
%% PDF-A
%%%%%%%%%%%%%%%%%%%%%%%%%%%%%%%%%%%%%%%%%%%%%%%%%%%%%%%%%%%%%%%%%%%%%%%%%%%%%%%


%%%%%%%%%%%%%%%%%%%%%%%%%%%%%%%%%%%%%%%% 
% define medatata
%%%%%%%%%%%%%%%%%%%%%%%%%%%%%%%%%%%%%%%% 
% \def\Title{\ttitle}
% \def\Author{\tauthor, matjaz.kralj@fri.uni-lj.si}
% \def\Subject{\ttitleEn}
% \def\Keywords{\tkeywordsEn}

%%%%%%%%%%%%%%%%%%%%%%%%%%%%%%%%%%%%%%%% 
% \convertDate converts D:20080419103507+02'00' to 2008-04-19T10:35:07+02:00
%%%%%%%%%%%%%%%%%%%%%%%%%%%%%%%%%%%%%%%% 
\def\convertDate{%
    \getYear
}

{\catcode`\D=12
 \gdef\getYear D:#1#2#3#4{\edef\xYear{#1#2#3#4}\getMonth}
}
\def\getMonth#1#2{\edef\xMonth{#1#2}\getDay}
\def\getDay#1#2{\edef\xDay{#1#2}\getHour}
\def\getHour#1#2{\edef\xHour{#1#2}\getMin}
\def\getMin#1#2{\edef\xMin{#1#2}\getSec}
\def\getSec#1#2{\edef\xSec{#1#2}\getTZh}
\def\getTZh +#1#2{\edef\xTZh{#1#2}\getTZm}
\def\getTZm '#1#2'{%
    \edef\xTZm{#1#2}%
    \edef\convDate{\xYear-\xMonth-\xDay T\xHour:\xMin:\xSec+\xTZh:\xTZm}%
}

\expandafter\convertDate\pdfcreationdate 

%%%%%%%%%%%%%%%%%%%%%%%%%%%%%%%%%%%%%%%%
% get pdftex version string
%%%%%%%%%%%%%%%%%%%%%%%%%%%%%%%%%%%%%%%% 
\newcount\countA
\countA=\pdftexversion
\advance \countA by -100
\def\pdftexVersionStr{pdfTeX-1.\the\countA.\pdftexrevision}


%%%%%%%%%%%%%%%%%%%%%%%%%%%%%%%%%%%%%%%%
% XMP data
%%%%%%%%%%%%%%%%%%%%%%%%%%%%%%%%%%%%%%%%  
% \usepackage{xmpincl}
% \includexmp{pdfa-1b}

%%%%%%%%%%%%%%%%%%%%%%%%%%%%%%%%%%%%%%%%
% pdfInfo
%%%%%%%%%%%%%%%%%%%%%%%%%%%%%%%%%%%%%%%%  
% \pdfinfo{%
%     /Title    (\ttitle)
%     /Author   (\tauthor, damjan@cvetan.si)
%     /Subject  (\ttitleEn)
%     /Keywords (\tkeywordsEn)
%     /ModDate  (\pdfcreationdate)
%     /Trapped  /False
% }


%%%%%%%%%%%%%%%%%%%%%%%%%%%%%%%%%%%%%%%%%%%%%%%%%%%%%%%%%%%%%%%%%%%%%%%%%%%%%%%
%%%%%%%%%%%%%%%%%%%%%%%%%%%%%%%%%%%%%%%%%%%%%%%%%%%%%%%%%%%%%%%%%%%%%%%%%%%%%%%







\begin{document}



\thispagestyle{empty}%
\begin{center}
 {\large\sc Univerza v Ljubljani\\%
   Fakulteta za računalništvo in informatiko}%
 \vskip 10em%
 {\autfont Tim Thuma (63230333), Tilen Medved (63230207), Luka Hribar (63230109) \par}%
 {\titfont Frodo's Nightmares \par}%
 {\vskip 3em \textsc{SEMINARSKA NALOGA\\[5mm]         % dodal Solina za ostale študijske programe
   VISOKOŠOLSKI STROKOVNI ŠTUDIJSKI PROGRAM\\ PRVE STOPNJE\\ RAČUNALNIŠTVO IN INFORMATIKA}\par}%
%  UNIVERZITETNI  ŠTUDIJSKI PROGRAM\\ PRVE STOPNJE\\ RAČUNALNIŠTVO IN INFORMATIKA}\par}%
%    INTERDISCIPLINARNI UNIVERZITETNI\\ ŠTUDIJSKI PROGRAM PRVE STOPNJE\\ RAČUNALNIŠTVO IN MATEMATIKA}\par}%
%    INTERDISCIPLINARNI UNIVERZITETNI\\ ŠTUDIJSKI PROGRAM PRVE STOPNJE\\ UPRAVNA INFORMATIKA}\par}%
%    INTERDISCIPLINARNI UNIVERZITETNI\\ ŠTUDIJSKI PROGRAM PRVE STOPNJE\\ MULTIMEDIJA}\par}%
 \vfill\null%
 {\large \textsc{Mentor}: izr. prof. dr. Iztok Lebar Bajec \par}%
 {\vskip 2em \large Ljubljana, januar 2025 \par}%
\end{center}



\newpage


\thispagestyle{empty}

% Abstract section
\section*{\textit{Abstract}}

\noindent Here you can write the abstract of your work. This should give a concise summary of your research, game design, and findings.

\newpage

% Pregled igre section
\section{Pregled igre}

\subsection{Opis sveta}
\noindent This section contains the description of the game world.

\subsection{Pregled}
\noindent Here you can describe an overview of the game.

\subsection{Ozadje}
\noindent This is the background of the game. You can insert an image here:

\begin{figure}[h!]
    \centering
    \includegraphics[width=0.5\textwidth]{./slika.png}
\end{figure}

\subsection{Ključne lokacije}
\noindent Describe the key locations within the game world here.

\subsection{Velikost}
\noindent Discuss the size of the game world or levels.

\subsection{Objekti}
\noindent Describe the main objects or elements in the game.

\newpage

% Igralni pogon section
\section{Igralni pogon}
\noindent This section describes the game engine used, its features, and how it supports the gameplay.

\newpage

% Pogled section
\section{Pogled}
\noindent This section describes the perspective or camera view in the game.

\newpage

% Osebek section
\section{Osebek}
\noindent Here you describe the protagonist or main character of the game.

\newpage

% Uporabniški vmesnik section
\section{Uporabniški vmesnik}
\noindent This section describes the user interface of the game, how the user interacts with it, and any menus or controls.

\newpage

% Gameplay section
\section{Gameplay}
\noindent Here you can describe the gameplay mechanics, objectives, challenges, and how players interact with the game world.

\newpage

% Zaključki in možne nadgradnje section
\section{Zaključki in možne nadgradnje}
\noindent Conclude your work here, and suggest possible improvements or upgrades for the game.

\end{document}
